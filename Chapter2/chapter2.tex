\documentclass{article}

\begin{document}
\section*{Exercises}

\begin{enumerate}
	\item \emph{Syntax}. Say whether each of the following is a sentence of Propositional Logic.
	\begin{enumerate}
		\item $p \wedge \neg p$

			Yes
		\item $\neg p \vee \neg p$

			Yes

		\item $\neg (q \vee r) \neg q \Rightarrow \neg\neg p$

			This is not a sentence in propositional logic. There is no operator between $\neg(q \vee r)$ and $\neg q$.

		\item $(p \vee q) \wedge (r \vee q)$

			Yes

		\item $p \vee \neg q \wedge \neg p \vee \neg q \Rightarrow p \vee q$

			Yes

		\item $( p \Rightarrow q) \vdash (q \Leftarrow p)$

			No, the $\vdash$ operator is not a valid operator in propositional logic.

		\item $(p \Rightarrow q) \models (q \Leftarrow p)$

			No, the $\models$ operator is not a valid operator in propositional logic.

		\item $((p \Rightarrow q) \Rightarrow s) \Leftrightarrow (r \Leftarrow t)$

			Yes.

		\item $((p \Leftrightarrow q) \Leftrightarrow s) \Leftrightarrow ( r \Leftrightarrow t)$

			Yes.

		\item This $\vee$ is $\neg$ correct.

			No, the $\neg$ operator is unary and there is no operator between \emph{This $\vee$ is} and \emph{$\neg$ correct}.
	\end{enumerate}
	\item \emph{Translation.} Consider a propositional language with three propositional constants -- \emph{purple, mushroom, poisonous} --
		each indicating the propert suggested by its spelling. Using these propositional constants, encode the following English sentences
		in propositional logic.


	\begin{enumerate}
		\item If a mushroom is purple, it is poisonous.

			P = Mushroom is purple

			T = Mushroom is poisonous

			$P \Rightarrow T$
	\end{enumerate}
\end{enumerate}
\end{document}